% Options for packages loaded elsewhere
\PassOptionsToPackage{unicode}{hyperref}
\PassOptionsToPackage{hyphens}{url}
%
\documentclass[
]{article}
\usepackage{amsmath,amssymb}
\usepackage{iftex}
\ifPDFTeX
  \usepackage[T1]{fontenc}
  \usepackage[utf8]{inputenc}
  \usepackage{textcomp} % provide euro and other symbols
\else % if luatex or xetex
  \usepackage{unicode-math} % this also loads fontspec
  \defaultfontfeatures{Scale=MatchLowercase}
  \defaultfontfeatures[\rmfamily]{Ligatures=TeX,Scale=1}
\fi
\usepackage{lmodern}
\ifPDFTeX\else
  % xetex/luatex font selection
\fi
% Use upquote if available, for straight quotes in verbatim environments
\IfFileExists{upquote.sty}{\usepackage{upquote}}{}
\IfFileExists{microtype.sty}{% use microtype if available
  \usepackage[]{microtype}
  \UseMicrotypeSet[protrusion]{basicmath} % disable protrusion for tt fonts
}{}
\makeatletter
\@ifundefined{KOMAClassName}{% if non-KOMA class
  \IfFileExists{parskip.sty}{%
    \usepackage{parskip}
  }{% else
    \setlength{\parindent}{0pt}
    \setlength{\parskip}{6pt plus 2pt minus 1pt}}
}{% if KOMA class
  \KOMAoptions{parskip=half}}
\makeatother
\usepackage{xcolor}
\usepackage[margin=1in]{geometry}
\usepackage{graphicx}
\makeatletter
\def\maxwidth{\ifdim\Gin@nat@width>\linewidth\linewidth\else\Gin@nat@width\fi}
\def\maxheight{\ifdim\Gin@nat@height>\textheight\textheight\else\Gin@nat@height\fi}
\makeatother
% Scale images if necessary, so that they will not overflow the page
% margins by default, and it is still possible to overwrite the defaults
% using explicit options in \includegraphics[width, height, ...]{}
\setkeys{Gin}{width=\maxwidth,height=\maxheight,keepaspectratio}
% Set default figure placement to htbp
\makeatletter
\def\fps@figure{htbp}
\makeatother
\setlength{\emergencystretch}{3em} % prevent overfull lines
\providecommand{\tightlist}{%
  \setlength{\itemsep}{0pt}\setlength{\parskip}{0pt}}
\setcounter{secnumdepth}{-\maxdimen} % remove section numbering
\ifLuaTeX
  \usepackage{selnolig}  % disable illegal ligatures
\fi
\IfFileExists{bookmark.sty}{\usepackage{bookmark}}{\usepackage{hyperref}}
\IfFileExists{xurl.sty}{\usepackage{xurl}}{} % add URL line breaks if available
\urlstyle{same}
\hypersetup{
  pdftitle={Guide MVE Bébé - Rapport Quantitatif},
  pdfauthor={Yannick Dufresne, Jérémy Gilbert, Hubert Cadieux, Benjamin Carignan},
  hidelinks,
  pdfcreator={LaTeX via pandoc}}

\title{Guide MVE Bébé - Rapport Quantitatif}
\author{Yannick Dufresne, Jérémy Gilbert, Hubert Cadieux, Benjamin
Carignan}
\date{2023-09-06}

\begin{document}
\maketitle

\newpage

\hypertarget{caractuxe9ristiques-du-ruxe9pondant}{%
\section{Caractéristiques du
répondant}\label{caractuxe9ristiques-du-ruxe9pondant}}

\includegraphics[width=0.7\textwidth,height=\textheight]{../Guide_mve_bebe/_SharedFolder_Guide_mve/graphs/ses_age.png}
\newline Nous ne retrouvons que très peu de répondants agés de 40 à 49
ans, aucun répondants de 50 ans ou plus, la majorité se trouvant dans
les deux autres tranches d'âge. Cependant, les 30 à 39 ans ont plus de
répondants que les 18-29 ans.

\includegraphics[width=0.7\textwidth,height=\textheight]{../Guide_mve_bebe/_SharedFolder_Guide_mve/graphs/ses_kidsLink.png}
\newline Nous remarquons que la majorité des répondants sont des mères,
avec toutefois un nombre intéressant de pères considérant que
l'échantillonnage souhaité peut avoir mené à cette distortion.

\newpage

\hypertarget{analyse-de-la-connaissance-et-de-lutilisation-du-guide}{%
\section{Analyse de la connaissance et de l'utilisation du
guide}\label{analyse-de-la-connaissance-et-de-lutilisation-du-guide}}

\begin{figure}[htbp]
  \centering
  \begin{minipage}{0.5\textwidth}
    \includegraphics[width=\linewidth]{../Guide_mve_bebe/_SharedFolder_Guide_mve/graphs/1guide_connaitre.png}
  \end{minipage}%
  \begin{minipage}{0.5\textwidth}
    \includegraphics[width=\linewidth]{../Guide_mve_bebe/_SharedFolder_Guide_mve/graphs/2guide_consulte.png}
  \end{minipage}
\end{figure}

\vspace{10pt}

\begin{minipage}{1.0\textwidth}
  \textbf{} 
Tel que présenté dans les graphiques ci-dessus, nous pouvons constater qu’une majorité des répondants au sondage connaissent le guide MVE, et de ce nombre, la majorité ont consulté le guide. Cela signifie que l’échantillon cible de ce sondage est bel et bien renseigné sur l’existence du guide en plus d'y trouver une utilité à sa consultation, tel que souhaité.

\end{minipage}

\vspace{30pt}

\includegraphics[width=0.9\textwidth,height=\textheight]{../Guide_mve_bebe/_SharedFolder_Guide_mve/graphs/12_connaitreXweeks.png}
\vspace{20pt} \newline Cependant, lorsque le niveau de connaissance est
analysé en fonction de l'avancement de la grossesse, il est possible de
constater une tendance à la hausse, bien que très légère.

\includegraphics[width=0.9\textwidth,height=\textheight]{../Guide_mve_bebe/_SharedFolder_Guide_mve/graphs/12_connaitreXweeksX1stkid.png}
\vspace{20pt} \newline Cette tendance à la hausse s'accentue davantage
lorsque le répondant est enceinte et qu'il est à son premier enfant. En
effet, il est probable qu'un répondant à sa première grossesse n'ait
aucune connaissance du guide, mais que les chances qu'il le connaisse
augmentent au fil des semaines de grossesses.

\includegraphics[width=0.75\textwidth,height=\textheight]{../Guide_mve_bebe/_SharedFolder_Guide_mve/graphs/ses_kids.png}

\begin{figure}[htbp]
  \centering
  \begin{minipage}{0.7\textwidth}
    \includegraphics[width=\linewidth]{../Guide_mve_bebe/_SharedFolder_Guide_mve/graphs/13_connaitreXkids}
  \end{minipage}%
  \begin{minipage}{0.3\textwidth}
    \textbf{}
    Plus spécifiquement, une augmentation du  nombres d'enfants correspond à une hausse du taux de connaissance du guide. Cependant, le niveau de connaissance du guide n'augmente que légèrement à partir du deuxième enfant.
  \end{minipage}
\end{figure}

\includegraphics[width=0.75\textwidth,height=\textheight]{../Guide_mve_bebe/_SharedFolder_Guide_mve/graphs/ses_immigrant.png}

\begin{figure}[htbp]
  \centering
  \begin{minipage}{0.7\textwidth}
    \includegraphics[width=\linewidth]{../Guide_mve_bebe/_SharedFolder_Guide_mve/graphs/13_connaitreXses_canada}
  \end{minipage}%
  \begin{minipage}{0.3\textwidth}
    \textbf{}Nous remarquons également que les répondants nées hors Canada ne connaissent généralement pas le guide, notamment lorsqu'ils sont plus agés. Plus l'âge d'un répondant immigrant est faible, plus que les probabilités de connaître le guide sont élevés. Le niveau de connaissance pour un répondant né au Canada est généralement plus élevé, à l'exception d'une petite baisse pour les plus jeunes.

  \end{minipage}
\end{figure}

\textbackslash begin\{figure\}{[}htbp{]} \centering

\begin{minipage}{0.7\textwidth}
    \includegraphics[width=\linewidth]{../Guide_mve_bebe/_SharedFolder_Guide_mve/graphs/13_connaitreXstatus.png}
  \end{minipage}

\% \textbackslash begin\{minipage\}\{0.3\textwidth\} \textbf{} Les
répondants avec un revenu considéré comme faible sont moins enclins à
connaitre le guide. Plus le revenu est élevé, plus la proportion de
connaissance est grande. Cette différence est largement accentuée par le
statut du parent. Les répondants enceintes pour la première fois sont
ceux qui connaissent moins le guide, et c'est encore plus vrai pour ceux
avec un faible revenu. Le niveau de connaissance est plus élevé et les
différences sont moins accentuées pour les parents enceintes mais déjà
parent. Finalement, les répondants non-enceintes et déjà parents
connaissent généralement tous le guide, avec des différences minimes en
fonction du revenu. \textbackslash end\{figure\}

\begin{figure}[htbp]
  \centering
  \begin{minipage}{0.7\textwidth}
    \includegraphics[width=\linewidth]{../Guide_mve_bebe/_SharedFolder_Guide_mve/graphs/Xconnaitre_age.png}
  \end{minipage}%
  \begin{minipage}{0.3\textwidth}
    \textbf{} La proportion de non-connaissance des répondants dans le groupe des 40 à 49 ans est légèrement plus élevé que les deux groupes plus jeunes. Cependant, il ne s'agit que d'une différence de quelques points de pourcentage. Cette différence n'est pas significative, mais elle est tout de même considérable.
  \end{minipage}
\end{figure}

\begin{figure}[htbp]
  \centering
  \begin{minipage}{0.8\textwidth}
    \includegraphics[width=\linewidth]{../Guide_mve_bebe/_SharedFolder_Guide_mve/graphs/Xconnaitre_educ.png}
  \end{minipage}%
  \begin{minipage}{0.2\textwidth}
    \textbf{} Il n'y a pas de tendance claire et significative.
  \end{minipage}
\end{figure}

\begin{figure}[htbp]
  \centering
  \begin{minipage}{0.9\textwidth}
    \includegraphics[width=\linewidth]{../Guide_mve_bebe/_SharedFolder_Guide_mve/graphs/Xconnaitre_region.png}
  \end{minipage}%
  \begin{minipage}{0.1\textwidth}
    \textbf{} Il n'y a pas de tendance claire, si ce n'est que l'Abitibi-Témiscaminque et Montréal ont une plus grande proportion de non connaissance.
  \end{minipage}
\end{figure}

\begin{figure}[htbp]
  \centering
  \begin{minipage}{0.7\textwidth}
    \includegraphics[width=\linewidth]{../Guide_mve_bebe/_SharedFolder_Guide_mve/graphs/Xutilise_age.png}
  \end{minipage}%
  \begin{minipage}{0.3\textwidth}
    \textbf{} Nous retrouvons une brève tendance dans l'utilisation du guide à la baisse lorsque l'âge augmente. Cependant, l'utilisation du guide est élevée dans tout les cas. En majorité ceux qui connaissent le guide l'utilisent.
  \end{minipage}
\end{figure}

\begin{figure}[htbp]
  \centering
  \begin{minipage}{0.8\textwidth}
    \includegraphics[width=\linewidth]{../Guide_mve_bebe/_SharedFolder_Guide_mve/graphs/Xutilise_educ.png}
  \end{minipage}%
  \begin{minipage}{0.2\textwidth}
    \textbf{} Il n'y a pas de tendance claire et significative.
  \end{minipage}
\end{figure}

\begin{figure}[htbp]
  \centering
  \begin{minipage}{0.9\textwidth}
    \includegraphics[width=\linewidth]{../Guide_mve_bebe/_SharedFolder_Guide_mve/graphs/Xutilise_region.png}
  \end{minipage}%
  \begin{minipage}{0.1\textwidth}
    \textbf{} Les taux d'utilisation par régions sont tous relativement faible. Le faible échantillon dans une région donnée explique le taux plus élevé de la Côte-Nord et de l'Outaouais.
  \end{minipage}
\end{figure}

\newpage

\hypertarget{format-papier-ou-format-web}{%
\section{Format papier ou format web
?}\label{format-papier-ou-format-web}}

\begin{figure}[htbp]
  \centering
  \begin{minipage}{0.7\textwidth}
    \includegraphics[width=\linewidth]{../Guide_mve_bebe/_SharedFolder_Guide_mve/graphs/3guide_format.png}
  \end{minipage}%
  \begin{minipage}{0.3\textwidth}
    \textbf{} Le guide est consulté en majorité dans son format papier. Seulement 30 répondants affirment utiliser uniquement le guide sur le web. Le format web est tout de même utilisé par un grande proportion des répondants, mais très peu n’utilisent pas du tout la version papier, ce qui laisse présager que le web seul ne convient pas à tous.
  \end{minipage}
\end{figure}
\newpage

\includegraphics[width=0.75\textwidth,height=\textheight]{../Guide_mve_bebe/_SharedFolder_Guide_mve/graphs/5freq.png}
\newline \includegraphics[width=0.75\textwidth,height=\textheight]{../Guide_mve_bebe/_SharedFolder_Guide_mve/graphs/6guide_prop_density.png}
\newline Il n'y a pas de différence significative au niveau de la
fréquence de consultation selon si le format papier ou web a été
utilisé. Cependant, il y a une grande différence entre le format papier
et le format web au niveau de la proportion lue du guide. En effet, les
répondants ayant lu le format papier ont généralement lu une plus grande
proportion du guide que ceux utilisant le format web. Nous retrouvons
119 répondants ayant lu le guide au complet, contre seulement 19 pour le
format web, ce qui représente une différence considérable entre les
proportions de consultations du guide. Au-dessus de 25\% des
utilisateurs du web ne lisent que quelques pages, alors que c'est moins
de 10\% pour ceux le lisant papier. Également, près de la moitié des
répondants qui disent utiliser le guide papier en lisent au moins les
trois quarts.

\includegraphics{../Guide_mve_bebe/_SharedFolder_Guide_mve/graphs/7guide_satisfactiondensity.png}
Le niveau de satisfaction quant au format du guide est généralement plus
élevé chez les utilisateurs du format papier que du format web. Tout
d'abord, nous ne retrouvons pratiquement pas de répondants insatisfaits
par le guide papier, alors que ceux utilisant le format web le sont dans
certains cas. Ensuite, la proportion d'utilisateurs du format web
atteint un niveau de satisfaction complet moins souvent que le format
papier. En bref, les utilisateurs du format papier considèrent que le
guide MVE remplit pleinement sa fonction.

\newpage
\begin{figure}[htbp]
  \centering
  \begin{minipage}{0.5\textwidth}
    \includegraphics[width=\linewidth]{../Guide_mve_bebe/_SharedFolder_Guide_mve/graphs/9guide_pratique.png}
  \end{minipage}%
  \begin{minipage}{0.5\textwidth}
    \includegraphics[width=\linewidth]{../Guide_mve_bebe/_SharedFolder_Guide_mve/graphs/10guide_desavantages.png}
  \end{minipage}
\end{figure}

\vspace{10pt}

\begin{minipage}{1.0\textwidth}
  \textbf{} 
Les tendances qui ressortent des analyses sont que le guide format papier est facile à consulter et que le guide format web est accessible en tout temps, en plus d’être facile à consulter. Le facilité de consultation du format papier concorde avec son niveau de satisfaction procuré. En effet, un guide facile à consulter est logiquement corrélé avec une appréciation plus élevée. À l'inverse, le guide papier aurait comme désavatange principal d'être difficile à transporter, alors que le format web peut être difficile d'accès mais surtout difficile à utiliser.
\end{minipage}

\vspace{10pt}

\includegraphics[width=0.8\textwidth,height=\textheight]{../Guide_mve_bebe/_SharedFolder_Guide_mve/graphs/11guide_format_change.png}
La difficulté de transport du guide dans son format papier semblerait
être une motivation suffisante pour un peu plus que la moitié de ceux
ayant répondu à la question.

\newpage

\hypertarget{conclusion---recommandations-en-fonction-des-analyses}{%
\section{Conclusion - Recommandations en fonction des
analyses}\label{conclusion---recommandations-en-fonction-des-analyses}}

\hypertarget{connaissance-et-utilisation-du-guide}{%
\subsection{Connaissance et utilisation du
guide}\label{connaissance-et-utilisation-du-guide}}

\begin{itemize}
\tightlist
\item
  Il est de règle générale que ceux qui connaissent le guide l'utilisent
  en grande partie. Cependant, il est possible de trouver quelques
  lacunes à corriger quant au niveau de connaissance du guide, bien
  qu'il soit assez élevé.
\item
  Le fait que les répondants enceintes pour la première fois connaissent
  moins le guide au début de leur grossesse est un constat intéressant.
  En effet, afin d'aider à la connaissance du guide, il serait pertinent
  de cibler les premières grossesses plus rapidement et de leur
  présenter le guide. De plus, comme le niveau de connaissance du guide
  augmente à chaque enfant, il serait intéressant de promouvoir
  l'échange du guide et d'informations entre les parents.
\item
  Nous remarquons également que les personnes plus âgées (40 à 49 ans)
  nées hors-Canada connaissent moins le guide que les plus jeunes. Il
  doit donc y avoir un effort pour rejoindre les immigrants plus âgés
  afin de leur présenter le guide. De manière générale, le niveau de
  connaissance du guide est bon pour les personnes nées au Canada, à
  l'exception des plus jeunes (18 à 29 ans). Pour eux, il sera important
  de les rejoindre et de leur présenter le guide, alors que les jeunes
  répondants nés hors-Canada connaissent très bien le guide.
\item
  Il est démontré plus haut qu'il y a des différences notables dans le
  niveau de connaissance du répondant en fonction de son statut de
  parent et de son niveau de revenu. Il est conseillé dans tout les cas
  de rejoindre davantage les personnes à plus faible revenu, car il
  connaissent moins le guide de manière générale. il est important aussi
  de se concentrer sur les personnes enceintes, surtout pour la première
  fois, car ils connaissent moins le guide que ceux déjà parents.
\end{itemize}

\hypertarget{format-papier-et-web.}{%
\subsection{Format papier et web.}\label{format-papier-et-web.}}

\begin{itemize}
\tightlist
\item
\end{itemize}

\end{document}
